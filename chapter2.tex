\documentclass[]{article}
\usepackage{lmodern}
\usepackage{amssymb,amsmath}
\usepackage{ifxetex,ifluatex}
\usepackage{fixltx2e} % provides \textsubscript
\ifnum 0\ifxetex 1\fi\ifluatex 1\fi=0 % if pdftex
  \usepackage[T1]{fontenc}
  \usepackage[utf8]{inputenc}
\else % if luatex or xelatex
  \ifxetex
    \usepackage{mathspec}
  \else
    \usepackage{fontspec}
  \fi
  \defaultfontfeatures{Ligatures=TeX,Scale=MatchLowercase}
\fi
% use upquote if available, for straight quotes in verbatim environments
\IfFileExists{upquote.sty}{\usepackage{upquote}}{}
% use microtype if available
\IfFileExists{microtype.sty}{%
\usepackage{microtype}
\UseMicrotypeSet[protrusion]{basicmath} % disable protrusion for tt fonts
}{}
\usepackage[margin=1in]{geometry}
\usepackage{hyperref}
\hypersetup{unicode=true,
            pdfborder={0 0 0},
            breaklinks=true}
\urlstyle{same}  % don't use monospace font for urls
\usepackage{color}
\usepackage{fancyvrb}
\newcommand{\VerbBar}{|}
\newcommand{\VERB}{\Verb[commandchars=\\\{\}]}
\DefineVerbatimEnvironment{Highlighting}{Verbatim}{commandchars=\\\{\}}
% Add ',fontsize=\small' for more characters per line
\usepackage{framed}
\definecolor{shadecolor}{RGB}{248,248,248}
\newenvironment{Shaded}{\begin{snugshade}}{\end{snugshade}}
\newcommand{\KeywordTok}[1]{\textcolor[rgb]{0.13,0.29,0.53}{\textbf{#1}}}
\newcommand{\DataTypeTok}[1]{\textcolor[rgb]{0.13,0.29,0.53}{#1}}
\newcommand{\DecValTok}[1]{\textcolor[rgb]{0.00,0.00,0.81}{#1}}
\newcommand{\BaseNTok}[1]{\textcolor[rgb]{0.00,0.00,0.81}{#1}}
\newcommand{\FloatTok}[1]{\textcolor[rgb]{0.00,0.00,0.81}{#1}}
\newcommand{\ConstantTok}[1]{\textcolor[rgb]{0.00,0.00,0.00}{#1}}
\newcommand{\CharTok}[1]{\textcolor[rgb]{0.31,0.60,0.02}{#1}}
\newcommand{\SpecialCharTok}[1]{\textcolor[rgb]{0.00,0.00,0.00}{#1}}
\newcommand{\StringTok}[1]{\textcolor[rgb]{0.31,0.60,0.02}{#1}}
\newcommand{\VerbatimStringTok}[1]{\textcolor[rgb]{0.31,0.60,0.02}{#1}}
\newcommand{\SpecialStringTok}[1]{\textcolor[rgb]{0.31,0.60,0.02}{#1}}
\newcommand{\ImportTok}[1]{#1}
\newcommand{\CommentTok}[1]{\textcolor[rgb]{0.56,0.35,0.01}{\textit{#1}}}
\newcommand{\DocumentationTok}[1]{\textcolor[rgb]{0.56,0.35,0.01}{\textbf{\textit{#1}}}}
\newcommand{\AnnotationTok}[1]{\textcolor[rgb]{0.56,0.35,0.01}{\textbf{\textit{#1}}}}
\newcommand{\CommentVarTok}[1]{\textcolor[rgb]{0.56,0.35,0.01}{\textbf{\textit{#1}}}}
\newcommand{\OtherTok}[1]{\textcolor[rgb]{0.56,0.35,0.01}{#1}}
\newcommand{\FunctionTok}[1]{\textcolor[rgb]{0.00,0.00,0.00}{#1}}
\newcommand{\VariableTok}[1]{\textcolor[rgb]{0.00,0.00,0.00}{#1}}
\newcommand{\ControlFlowTok}[1]{\textcolor[rgb]{0.13,0.29,0.53}{\textbf{#1}}}
\newcommand{\OperatorTok}[1]{\textcolor[rgb]{0.81,0.36,0.00}{\textbf{#1}}}
\newcommand{\BuiltInTok}[1]{#1}
\newcommand{\ExtensionTok}[1]{#1}
\newcommand{\PreprocessorTok}[1]{\textcolor[rgb]{0.56,0.35,0.01}{\textit{#1}}}
\newcommand{\AttributeTok}[1]{\textcolor[rgb]{0.77,0.63,0.00}{#1}}
\newcommand{\RegionMarkerTok}[1]{#1}
\newcommand{\InformationTok}[1]{\textcolor[rgb]{0.56,0.35,0.01}{\textbf{\textit{#1}}}}
\newcommand{\WarningTok}[1]{\textcolor[rgb]{0.56,0.35,0.01}{\textbf{\textit{#1}}}}
\newcommand{\AlertTok}[1]{\textcolor[rgb]{0.94,0.16,0.16}{#1}}
\newcommand{\ErrorTok}[1]{\textcolor[rgb]{0.64,0.00,0.00}{\textbf{#1}}}
\newcommand{\NormalTok}[1]{#1}
\usepackage{graphicx,grffile}
\makeatletter
\def\maxwidth{\ifdim\Gin@nat@width>\linewidth\linewidth\else\Gin@nat@width\fi}
\def\maxheight{\ifdim\Gin@nat@height>\textheight\textheight\else\Gin@nat@height\fi}
\makeatother
% Scale images if necessary, so that they will not overflow the page
% margins by default, and it is still possible to overwrite the defaults
% using explicit options in \includegraphics[width, height, ...]{}
\setkeys{Gin}{width=\maxwidth,height=\maxheight,keepaspectratio}
\IfFileExists{parskip.sty}{%
\usepackage{parskip}
}{% else
\setlength{\parindent}{0pt}
\setlength{\parskip}{6pt plus 2pt minus 1pt}
}
\setlength{\emergencystretch}{3em}  % prevent overfull lines
\providecommand{\tightlist}{%
  \setlength{\itemsep}{0pt}\setlength{\parskip}{0pt}}
\setcounter{secnumdepth}{0}
% Redefines (sub)paragraphs to behave more like sections
\ifx\paragraph\undefined\else
\let\oldparagraph\paragraph
\renewcommand{\paragraph}[1]{\oldparagraph{#1}\mbox{}}
\fi
\ifx\subparagraph\undefined\else
\let\oldsubparagraph\subparagraph
\renewcommand{\subparagraph}[1]{\oldsubparagraph{#1}\mbox{}}
\fi

%%% Use protect on footnotes to avoid problems with footnotes in titles
\let\rmarkdownfootnote\footnote%
\def\footnote{\protect\rmarkdownfootnote}

%%% Change title format to be more compact
\usepackage{titling}

% Create subtitle command for use in maketitle
\newcommand{\subtitle}[1]{
  \posttitle{
    \begin{center}\large#1\end{center}
    }
}

\setlength{\droptitle}{-2em}

  \title{}
    \pretitle{\vspace{\droptitle}}
  \posttitle{}
    \author{}
    \preauthor{}\postauthor{}
    \date{}
    \predate{}\postdate{}
  

\begin{document}

\section{Regression and model
validation}\label{regression-and-model-validation}

Week 2

The data used is from a study which investigates the relationship
between teaching, studying habits and results of the students.

First we import the data which we wrangled and saved locally in the
first part of the exercise. The data is set to a variable \texttt{lrn14}
using read.table.

\begin{Shaded}
\begin{Highlighting}[]
\NormalTok{lrn14 <-}\StringTok{ }\KeywordTok{read.table}\NormalTok{(}\StringTok{"C:/Users/Anton/Documents/IODS-project/learning2014.txt"}\NormalTok{)}
\end{Highlighting}
\end{Shaded}

We also need to import few dependancies for later use:

\begin{Shaded}
\begin{Highlighting}[]
\CommentTok{# library(dplyr)}
\KeywordTok{library}\NormalTok{(GGally)}
\end{Highlighting}
\end{Shaded}

\begin{verbatim}
## Loading required package: ggplot2
\end{verbatim}

\begin{Shaded}
\begin{Highlighting}[]
\KeywordTok{library}\NormalTok{(ggplot2)}
\end{Highlighting}
\end{Shaded}

By using str() we can have a look at the structure of the data. The
output gives us meta information about the data in general and about
each column itself: amount of rows and colums, name of the variables,
data type and the first values.

\begin{Shaded}
\begin{Highlighting}[]
\KeywordTok{str}\NormalTok{(lrn14)}
\end{Highlighting}
\end{Shaded}

\begin{verbatim}
## 'data.frame':    166 obs. of  7 variables:
##  $ gender  : Factor w/ 2 levels "F","M": 1 2 1 2 2 1 2 1 2 1 ...
##  $ Age     : int  53 55 49 53 49 38 50 37 37 42 ...
##  $ Attitude: int  37 31 25 35 37 38 35 29 38 21 ...
##  $ deep    : num  3.58 2.92 3.5 3.5 3.67 ...
##  $ stra    : num  3.38 2.75 3.62 3.12 3.62 ...
##  $ surf    : num  2.58 3.17 2.25 2.25 2.83 ...
##  $ Points  : int  25 12 24 10 22 21 21 31 24 26 ...
\end{verbatim}

Using \texttt{dim()} we get an output which gives as the dimensions of
the dataset.

\begin{Shaded}
\begin{Highlighting}[]
\KeywordTok{dim}\NormalTok{(lrn14)}
\end{Highlighting}
\end{Shaded}

\begin{verbatim}
## [1] 166   7
\end{verbatim}

Using this dataset we generate a graphical overview:

\begin{Shaded}
\begin{Highlighting}[]
\NormalTok{p <-}\StringTok{ }\KeywordTok{ggpairs}\NormalTok{(lrn14, }\DataTypeTok{mapping =} \KeywordTok{aes}\NormalTok{(), }\DataTypeTok{lower =} \KeywordTok{list}\NormalTok{(}\DataTypeTok{combo =} \KeywordTok{wrap}\NormalTok{(}\StringTok{"facethist"}\NormalTok{, }\DataTypeTok{bins =} \DecValTok{20}\NormalTok{)))}
\NormalTok{p}
\end{Highlighting}
\end{Shaded}

\includegraphics{chapter2_files/figure-latex/unnamed-chunk-5-1.pdf}

And provide a summary:

\begin{Shaded}
\begin{Highlighting}[]
\KeywordTok{summary}\NormalTok{(lrn14)}
\end{Highlighting}
\end{Shaded}

\begin{verbatim}
##  gender       Age           Attitude          deep            stra      
##  F:110   Min.   :17.00   Min.   :14.00   Min.   :1.583   Min.   :1.250  
##  M: 56   1st Qu.:21.00   1st Qu.:26.00   1st Qu.:3.333   1st Qu.:2.625  
##          Median :22.00   Median :32.00   Median :3.667   Median :3.188  
##          Mean   :25.51   Mean   :31.43   Mean   :3.680   Mean   :3.121  
##          3rd Qu.:27.00   3rd Qu.:37.00   3rd Qu.:4.083   3rd Qu.:3.625  
##          Max.   :55.00   Max.   :50.00   Max.   :4.917   Max.   :5.000  
##       surf           Points     
##  Min.   :1.583   Min.   : 7.00  
##  1st Qu.:2.417   1st Qu.:19.00  
##  Median :2.833   Median :23.00  
##  Mean   :2.787   Mean   :22.72  
##  3rd Qu.:3.167   3rd Qu.:27.75  
##  Max.   :4.333   Max.   :33.00
\end{verbatim}


\end{document}
